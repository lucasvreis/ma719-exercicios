\documentclass[../main.tex]{subfiles}

\begin{document}
    \subsection*{9.2.12}
    Seja $K=\F^m$ e considere \(T\in\Hom_\F(V,K)\) dado por $T(v)=\sum_{i=1}^m f_i(v)e_i$, onde \(\alpha=\{e_i\}_{i=1}^m\) é a base canônica de $K$. Claro que $W=\bigcap_{j=1}^m\Ker(f_j)=\Ker(T)$. Por isso, $V/W\cong{T(V)}\subset K$, e segue que $\dim(V/W)\leq\dim(K)=m$. Além disso, vimos no exercício 9.2.5 que se $\{f_i\}_{i=1}^m$ for l.i., então $T(V)=K$, e reciprocamente, se $\dim(V/W)=m$ então $T$ é sobrejetora; nesse caso, considere \(a_i\in\F\) tal que $a_1f_1+\dots+a_mf_m=0$. Podemos definir um funcional $g\in K^*$ por $g(e_i)=a_i$ para $1\leq{i}\leq{m}$, e dessa forma:
    \[(T^tg)(v)=g\left(\sum_{i=1}^m f_i(v)e_i\right)=\sum_{i=1}^m g(e_i)f_i(v)=\sum_{i=1}^m a_if_i(v)=0\]
   Mas como $T$ é sobrejetora, para todo $1\leq i \leq m$ existe $v_i\in V$ tal que $T(v_i)=e_i$, de forma que $a_i=g(e_i)=(T^tg)(v_i)=0$, ou seja, $\{f_i\}_{i=1}^m$ é l.i..
\end{document}