\documentclass[../main.tex]{subfiles}

\begin{document}
Demonstração do corolário: se $V_{p}\neq\{0\}$, então deve existir um fator primo $f$ de $p$ com $V_{f}\neq\{0\}$, pelo seguinte motivo: suponha que os núcleos de $f\text{ e }g$ são triviais. Suponha $V_{fg}\ni v\neq0$, ou seja $(fg)(T)(v)=0$, mas isso implicaria $v\in V_{g}$ ou $0\neq g(T)(v)\in V_{f}$, contradição. Por outro lado, se $f\in\Div(m_{T})$, então $\{0\}\neq V_{f}\subset V_{fg}$. Por fim, se $g$ não tem fatores primos em $\Div(m_{T})$, isso significa que o núcleo de todos os seus fatores primos é $\{0\}$. Como vimos anteriormente, isso significa que o núcleo de $g$, que é o produto dos fatores, também é $\{0\}$. Mas então para todo $v\in V_{p}$, $(gf)(T)(v)=0$, o que implica $f(T)(v)\in V_{g}=\{0\}$, portanto $v\in V_{f}$. Logo $V_{f}\subset V_{p}\supset V_{f}$. Na segunda parte, sejam $f_{i}$ os divisores de $p$ com multiplicidade $k_{i}$. Sabemos que podemos tomar um vetor não nulo em $v_{i}\in V_{f_{i}^{k_{i}}}\setminus V_{f_{i}^{k_{i}-1}}$ pois um está contido estritamente no outro. Além disso, $m_{v_{i}}=f_{i}^{k_{i}}$ pelo seguinte motivo: $m_{v_{i}}|f_{i}^{k_{i}}$ já que $v_{i}\in V_{f_{i}^{k_{i}}}$, o que implica $m_{v_{i}}=f^{k'}$ com $k'\leq k_{i}$, mas para todo $k'<k_{i}, v_{i}\notin V_{f_{i}^{k'}}$, portanto $k'=k_{i}$. Seja $v=\sum_{i}v_{i}$, de forma que temos $m_{v}\mid\prod_{i}m_{v_{i}}$. Por contradição, suponha que $\prod_{i}m_{v_{i}}\nmid m_{v}$. Então existe um $f_{j}^{k_{j}}\nmid m_{\sum_{i}v_{i}}$, e nesse caso $m_{v}(T)(v)=\sum_{i}m_{v}(T)(v_{i})\neq0$ pois sempre que polinômios são coprimos, a restrição de um ao núcleo de outro é injetora, e os $m_{v}(T)(v_{i})$ moram em subespaços diferentes para $i$ distintos, de forma que a soma de todos é não nula pois por hipótese pelo menos $m_{v}(T)(v_{j})\neq0$. Ou seja, $m_{v}(T)(v)\neq0$, contradição. 
\end{document}