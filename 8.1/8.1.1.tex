\documentclass[../main.tex]{subfiles}

\begin{document}
Como hipótese do exercício, suponha que \(V_{p^{m+1}}=V_{p^m}\)
para algum \(m\geq 0\), e por hipótese de indução seja \(k\in\Z_{\geq 1}\)
tal que \(V_{p^m}=V_{p^{m+k}}\). Seja $v\in V$ um vetor qualquer.
Pelas hipóteses acima,
\[p^m(T)(v)=0\iff p^{m+1}(T)(v)=0\tag{1}\]
\[p^m(T)(p(T)(v))=0\iff p^{m+k}(T)(p(T)(v))=0.\tag{2}\]
Mas:
 \[p^m(T)(p(T)(v))=p^{m+1}(T)(v)\] \[p^{m+k}(T)(p(T)(v))=
p^{m+k+1}(T)(v)\]
assim, o lado direito de (1) é igual ao lado esquerdo de (2), e portanto
\[p^{m+k+1}(T)(v)=0\iff p^m(T)(v)=0\]
como queríamos. Para provar (8.1.5), note que como \(V_{p^i}
\subseteq V_{p^{i+1}}\) e a dimensão de \(V\) é finita, \(\dim(V_{p^i})\leq
\dim(V_{p^{i+1}})\) e vale a igualdade se, e somente se, 
\(V_{p^i}=V_{p^{i+1}}\). Assim, 
\[V_{p^0}\subseteq V_p\subseteq V_{p^2}\subseteq \dots\subseteq V_{p^n}\]
deve ser tal que \(\dim(V_{p^m})=\dim(V_{p^{m+1}})\) para algum \(0\leq
m\leq n\), caso contrário teríamos \(\dim(V_{p^{n+1}})> n=\dim(V)\),
um absurdo. Usando o resultado acima, concluímos que \(V_p^\infty=
V_{p^n}=V_{p^m}\).
\end{document}